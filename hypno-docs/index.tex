\documentclass[a4paper,12pt]{article}
\usepackage[T1]{fontenc}
\usepackage{ninecolors}
\usepackage{booktabs}
\usepackage{caption}
\usepackage{tabularray}
\usepackage{hyperref}
\usepackage{graphicx}
\usepackage{subcaption}
\usepackage{parskip}
\usepackage{tikz}
\usepackage{circuitikz}
\usepackage[tocentry]{vhistory}
\usepackage{float}
\hypersetup{
  colorlinks=true,
  linkcolor=blue,
  filecolor=magenta,
  urlcolor=cyan,
  pdftitle={Hypno},
  pdfpagemode=FullScreen,
}
\graphicspath{ {img/} }
\captionsetup[table]{position=bottom}
\usepackage{geometry}
\usepackage{siunitx}
\usepackage{awesomebox}

\tikzset{
  padStyle/.style={line width=1mm, draw=orange, fill=none}
}

\tikzset{
  partStyle/.style={line width=1mm, draw=black, fill=none, rounded corners=4pt}
}

\begin{document}

\begin{titlepage}
  \vspace*{\stretch{1.0}}
  \begin{center}
    \Large\textbf{Hypno}\\
    \large{Delay Kit by Pedal Markt}
  \end{center}
  \vspace*{\fill}
  \begin{center}
    \vhCurrentDate\\
    { Rev}\vhCurrentVersion
  \end{center}
\end{titlepage}

\tableofcontents
\pagebreak

\section{Introduction}

TODO

Enclosures for Hypno and other pedals in the Beastly Series
were designed by \href{https://fiz.gallery/}{Agata Fiz.}

\begin{figure}[h!]
  \centering
  \begin{subfigure}[b]{0.49\textwidth}
    \centering
    \includegraphics[width=\textwidth]{hypno-outside-1000px.jpg}
  \end{subfigure}
  \begin{subfigure}[b]{0.49\textwidth}
    \centering
    \includegraphics[width=\textwidth]{hypno-inside-1000px.jpg}
  \end{subfigure}
  \caption{Hypno: oustide and inside}
  \label{fig:Hypno}
\end{figure}


\pagebreak

\section{BOM – Bill of Materials}

BOM is a document that lists the parts you'd need to build a
project. Each row corresponds to a component with a certain
value, for example, a `ceramic capacitor with value 1nF.`
There could be one or more actual physical parts per
row, their designators are listed in the \textit{Reference}
column.

% \tipbox{
%   Components in the BOM are listed in order of assembly. Go
%   through the table top to bottom. If you haven't built a
%   kit before, check out the
%   \hyperref[sec:steps]{Step-by-step Instructions} first.
% }

\tipbox{
  In the BOM \textit{text in italic font} gives tips about how to mount or
  solder parts.
}

\newgeometry{hmargin={1cm}}

\begin{longtblr}[caption = {BOM}]{
  hlines,
  vlines,
  rows={ht=1.2em},
  row{even}={bg=gray9},
  row{1}={bg=gray3,fg=white},
  width=\linewidth,
  colspec={lX[1]llX[2]},
}
  \hspace{1em}
  & \textbf{Ref}
  & \textbf{Value}
  & \textbf{Qnty}
  & \textbf{Description}
  \\
  \SetCell[c=5]{c,bg=gray6,fg=white}\textbf{Outboard}
  \\
  \hspace{1em}
  & – & Enclosure & 1 & \textit{Mount the DC
  jack, Footswitch and Lampshade into the enclosure
  before soldering}
  \\
  \hspace{1em}
  & -- & Lampshade & 1
  & Small transparent plastic part for the LED,
  \textit{mount in enclosure before putting the boards in}
  \\
  \hspace{1em}
  & -- & Rubber Ring & 1
  & \textit{Use it to keep Lampshade in place}
  \\
  \hspace{1em}
  & -- & DC Jack & 1
  & Black plastic part with a nut, \textit{mount in
  enclosure before soldering}
  \\
  \hspace{1em}
  & -- & DC Cable & 1
  & Red and black cables in a JST connector, \textit{cut to
  $\approx10cm$ and solder to DC Jack once it's mounted in
  enclosure. Black wire to shorter lug, red to the longer
  one.}
  \\
  \hspace{1em}
  & -- & Audio Jack & 2 & \textit{Only mount these in the
  enclosure together with the main board once they are wired up}
  \\
  \SetCell[c=5]{c,bg=gray6,fg=white}\textbf{Main board, floor side}
  \\
  \hspace{1em}
  & GND & Wire & 2 & $\approx5cm$, black, \textit{strip and
  tin the ends}
  \\
  \hspace{1em}
  & IN & Wire & 1 & $\approx5cm$, any color, \textit{strip and tin
  the ends}
  \\
  \hspace{1em}
  & OUT & Wire & 1 & $\approx5cm$, any other color,
  \textit{strip and tin the ends}
  \\
  \hspace{1em}
  & R8 & 5.6k & 1
  & Resistor
  \\
  \hspace{1em}
  & R9 & 56k & 1 & Resistor
  \\
  \hspace{1em}
  & R10 & 100 & 1 & Resistor
  \\
  \hspace{1em}
  & R11 & 68k & 1 & Resistor
  \\
  \hspace{1em}
  & R12 & 100k & 1 & Resistor
  \\
  \hspace{1em}
  & R19 & 470k & 1 & Resistor
  \\
  \hspace{1em}
  & R20 & 1k & 1 & Resistor
  \\
  \hspace{1em}
  & R21 & 6.8k & 1 & Resistor
  \\
  \hspace{1em}
  & R2, R7 & 15k & 2 & Resistor
  \\
  \hspace{1em}
  & R13, R14 & 1M & 2 & Resistor
  \\
  \hspace{1em}
  & R15, R16, R17, R18 & 140k & 4 & Resistor
  \\
  \hspace{1em}
  & R1, R3, R4, R5, R6, R25, R26, R27 & 10k & 8 & Resistor
  \\
  \hspace{1em}
  & D3 & 1N4148 & 1 & Diode, \textit{Orientation matters}
  \\
  \hspace{1em}
  & D4 & 1N4001 & 1 & Diode, \textit{Orientation matters}
  \\
  \hspace{1em} & U2 & PT2399 & 1 & Delay chip. \textit{Please
  use socket. Orientation matters}
  \\
  \hspace{1em} & U3 & TL072 & 1 & Dual opamp. \textit{Please
  use socket. Orientation matters}
  \\
  \hspace{1em}
  & C18 & 1u & 1
  & Ceramic capacitor
  \\
  \hspace{1em}
  & C20 & 2.2p & 1
  & Ceramic capacitor
  \\
  \hspace{1em}
  & C7, C13 & 560p & 2
  & Ceramic capacitor
  \\
  \hspace{1em}
  & C15, C24 & 100n & 2
  & Ceramic capacitor
  \\
  \hspace{1em}
  & J1 & Power Socket & 1
  & JST 2-pin m, in the bottom-left part of the board,
  \textit{orientation matters}
  \\
  \hspace{1em}
  & Q1, Q2 & 2N3904 & 2
  & NPN transistor, \textit{orientation matters}
  \\
  \hspace{1em}
  & U1 & L78L05 & 1
  & Voltage regulator, \textit{orientation matters}
  \\
  \hspace{1em}
  & C14 & 10n & 1
  & Film capacitor
  \\
  \hspace{1em}
  & C5, C11 & 3.9n & 2
  & Film capacitor
  \\
  \hspace{1em}
  & C8, C9, C10, C12, C19 & 100n & 5
  & Film capacitor
  \\
  \hspace{1em}
  & C1, C17 & 1u & 2
  & Film capacitor
  \\
  \hspace{1em}
  & C6 & 100n & 1
  & Electrolytic capacitor, \textit{orientation matters}
  \\
  \hspace{1em}
  & C3 & 330n & 1
  & Electrolytic capacitor, \textit{orientation matters}
  \\
  \hspace{1em}
  & C26 & 100u & 1
  & Electrolytic capacitor, \textit{orientation matters}
  \\
  \hspace{1em}
  & C16, C25 & 10u & 2
  & Electrolytic capacitor, \textit{orientation matters}
  \\
  \hspace{1em}
  & C2, C4, C23 & 4.7u & 3
  & Electrolytic capacitor, \textit{orientation matters}
  \\
  \SetCell[c=5]{c,bg=gray6,fg=white}\textbf{Main board, player side}
  \\
  \hspace{1em}
  & -- & Ribbon cable & 1
  & Pads for that cable are in the bottom-center of the main
  board, \textit{solder one end to main board, another to
  switch board, \textbf{make sure pin names on the two
  boards match, IN on one board is connected to IN on the
  other board etc}}
  \\
  \hspace{1em}
  & RV1 (Sink) & C10k & 1
  & Potentiometer
  \\
  \hspace{1em}
  & RV2 (Recall) & A100k & 1
  & Potentiometer
  \\
  \hspace{1em}
  & RV3 (Lucidity) & B100k & 1
  & Potentiometer
  \\
  \SetCell[c=5]{c,bg=gray6,fg=white}\textbf{Switch board, player side}
  \\
  \hspace{1em}
  & Rled & 1k & 1
  & \textit{larger value will make the LED
  dimmer, values up to 6.8k are reasonable}
  \\
  \hspace{1em}
  & -- & LED & 1
  & \textit{Insert in PCB first. Solder last, once the
  main board is in the enclosure. Orientation matters}
  \\
  \hspace{1em}
  & -- & Footswitch & 1
  & \textit{Mount in enclosure before putting the boards in}
  \\
\end{longtblr}

\restoregeometry{}

\subsection{Note on values}

Different kits and schematics designate values differently.
For example, these usually mean the same value:
\\
$\SI{2.2}{\kohm} = 2.2k = 2k2 = 2.2 \times 10^{3} Ohm = 2200Ohm$
\\
$\SI{4.7}{\uF} = 4.7u = 4u7 = 4.7 \times 10^{-6} Farad = 0.0000047 Farad$

\begin{table}[h!]
  \caption{Component values}
  \centerline{
    \begin{tblr}{
      hlines,
      vlines,
      rows={ht=1.2em},
      row{1}={bg=gray3,fg=white},
      colspec={Xrr}
    }
      \textbf{Value}
      & \textbf{Multiplier}
      & \textbf{Unit}
      \\
      \SetCell[c=3]{c}\textbf{Resistance}
      \\
      \SI{100}{\ohm}, 100R, 100 & 1 & Ohm
      \\
      \SI{1}{\kohm}, 1k & $10^{3}$ & Ohm
      \\
      \SI{1}{\Mohm}, 1M & $10^{6}$ & Ohm
      \\
      \SetCell[c=3]{c}\textbf{Capacitance}
      \\
      \SI{1}{\pF}, 1p & $10^{-12}$ & Farad
      \\
      \SI{1}{\nF}, 1n & $10^{-9}$ & Farad
      \\
      \SI{1}{\uF}, 1u & $10^{-6}$ & Farad
    \end{tblr}
  }
\end{table}

\pagebreak

% \begin{figure}[h!]
%   \begin{center}
%     \includegraphics[width=0.5\textwidth]{bc108-pinout.png}
%   \end{center}
%   \caption{BC108 pinout}
% \end{figure}



% \section{Step-by-step Instructions}
% \label{sec:steps}

% \section{Schematic}
% \label{sec:schematic}

% \begin{figure}[h!]
%   \centering
%   \includegraphics[width=2\textwidth, trim={0 4cm 0cm 0cm}, page=2, clip, angle=-90]{schem.pdf}
% \end{figure}

% \newgeometry{}
% \restoregeometry{}

\begin{versionhistory}
  \vhEntry{1.0}{May 11, 2025}{AS}{Created}
\end{versionhistory}

\end{document}
