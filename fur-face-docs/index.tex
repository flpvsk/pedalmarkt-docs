\documentclass[a4paper,12pt]{article}
\usepackage[T1]{fontenc}
\usepackage{ninecolors}
\usepackage{booktabs}
\usepackage{caption}
\usepackage{tabularray}
% \usepackage[margin=2cm]{geometry}
\captionsetup[table]{position=bottom}

\begin{document}
\begin{titlepage}
  \vspace*{\stretch{1.0}}
  \begin{center}
    \Large\textbf{Build a Fur Face}\\
    \large{Soldering Workshop by Pedal Markt}
  \end{center}
  \vspace*{\fill}
  \begin{center}
    \today
  \end{center}
\end{titlepage}
\section{BOM – Bill of Materials}

Here is a list of parts you'd need to build the pedal.
Each row corresponds to a component with a certain value,
for example a `ceramic capacitor with value 1nF.` There could
be one or more actual physical parts per each row,
their designators are listed in the \textit{Reference}
column.

\begin{table}[h!]
  \centerline{
    \begin{tblr}{
      hlines,
      vlines,
      rows={2em},
      row{odd}={bg=gray9},
      width=1.3\linewidth,
      colspec={llllX}
    }
      \hspace{1em}
      & \textbf{Ref}
      & \textbf{Value}
      & \textbf{Qnty}
      & \textbf{Description}
      \\
      \hspace{1em}
      & H7 & IN & 1 & Wire
      \\
      \hspace{1em}
      & H14 & OUT & 1 & Wire
      \\
      \hspace{1em}
      & H8, H10 & GND & 2 & Wire
      \\
      \hspace{1em}
      & R2 & Rled, 330-10K & 1
      & Resistor, the larger the value the dimmer the LED
      \\
      \hspace{1em}
      & R1 & 100K & 1 & Resistor
      \\
      \hspace{1em}
      & R3 & 470 & 1 & Resistor
      \\
      \hspace{1em}
      & Q1, Q2 & NPN Transistor & 2
      & Pinout: emitter / base / collector, please use socket
      \\
      \hspace{1em}
      & J1 & Power Socket & 1
      & JST 2-pin m, in the top-center part of the board
    \end{tblr}
  }
  \caption{BOM}
\end{table}

\end{document}
