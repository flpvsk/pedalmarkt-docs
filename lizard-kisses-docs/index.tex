\documentclass[a4paper,12pt]{article}
\usepackage[T1]{fontenc}
\usepackage{ninecolors}
\usepackage{booktabs}
\usepackage{caption}
\usepackage{tabularray}
\usepackage{geometry}
\usepackage{siunitx}

\begin{document}
\begin{titlepage}
  \vspace*{\stretch{1.0}}
  \begin{center}
    \Large\textbf{Build Lizard Kisses Overdrive}\\
    \large{Soldering Workshop by Pedal Markt}
  \end{center}
  \vspace*{\fill}
  \begin{center}
    \today
  \end{center}
\end{titlepage}

\section{BOM – Bill of Materials}

BOM is a document that lists the parts you'd need to build a
project. Each row corresponds to a component with a certain
value, for example a `ceramic capacitor with value 1nF.`
There could be one or more actual physical part per each
row, their designators are listed in the \textit{Reference}
column.

You could experiment with some of the parts that have a
drastic effect on the sound of the pedal. For those parts
the BOM suggests using sockets. Possible components that
could go into those sockets are listed later in the section.

\newgeometry{top=1cm}

\begin{table}[h!]
  \caption{BOM}
  \centerline{
    \begin{tblr}{
      hlines,
      vlines,
      rows={ht=1.2em},
      row{odd}={bg=gray9},
      row{1}={bg=gray3,fg=white},
      width=1.3\linewidth,
      colspec={lX[1]llX[2]}
    }
      \hspace{1em}
      & \textbf{Ref}
      & \textbf{Value}
      & \textbf{Qnty}
      & \textbf{Description}
      \\
      \SetCell[c=5]{c,bg=gray3,fg=white}\textbf{Main board, floor side}
      \\
      \hspace{1em}
      & GND & Wire & 2 & $\approx10cm$, strip and pre-tin both ends
      \\
      \hspace{1em}
      & IN & Wire & 1 & $\approx10cm$, strip and pre-tin both ends
      \\
      \hspace{1em}
      & OUT & Wire & 1 & $\approx10cm$, strip and pre-tin both ends
      \\
      \hspace{1em}
      & D2 & 1N4148 & 1
      & Diode
      \\
      \hspace{1em}
      & R7 & 4.7k & 1 & Resistor
      \\
      \hspace{1em}
      & R1 & 1k & 1
      & Resistor for the LED
      \\
      \hspace{1em}
      & R12 & 1k & 1
      & Resistor
      \\
      \hspace{1em}
      & R13 & 20k & 1
      & Resistor
      \\
      \hspace{1em}
      & R6, R10 & 2.2k & 2
      & Resistor
      \\
      \hspace{1em}
      & R2, R5, R8 & 1M & 3
      & Resistor
      \\
      \hspace{1em}
      & R3, R4, R9, R11, R14, R15, R16 & 10k & 7
      & Resistor
      \\
      \hspace{1em}
      & Q2, Q3 & 3-pin socket  & 2
      & For transistor
      \\
      \hspace{1em}
      & C6, C7 & 2-pin socket & 2
      & For capacitor
      \\
      \hspace{1em}
      & Diode Pairs & 4-pin socket & 4
      & For clipping diodes
      \\
      \hspace{1em}
      & Q1 & TP0606 & 1
      & P-channel MOSFET
      \\
      \hspace{1em}
      & Q5 & 2N3906 & 1
      & PNP transistor
      \\
      \hspace{1em}
      & Q4, Q6 & 2N3904 & 2
      & NPN transistor
      \\
      \hspace{1em}
      & C5, C8 & 47p & 2
      & Ceramic capacitor
      \\
      \hspace{1em}
      & C3 & 47n & 1
      & Film capacitor
      \\
      \hspace{1em}
      & C9 & 100n & 1
      & Film capacitor
      \\
      \hspace{1em}
      & -- & Power Socket & 1
      & 2-pin JST on the bottom-left of the board
      \\
      \hspace{1em}
      & C4, C10, C11 & 1u & 3
      & Film capacitor
      \\
      \hspace{1em}
      & C1 & 100u & 1
      & Electrolytic capacitor
      \\
      \hspace{1em}
      & C2 & 47u & 1
      & Electrolytic capacitor
      \\
      \SetCell[c=5]{c,bg=gray3,fg=white}\textbf{Outboard}
      \\
      \hspace{1em}
      & -- & DC Jack & 1
      & Mount and wire up the DC jack
      \\
      \hspace{1em}
      & -- & Audio Jack & 2
      & Wire up audio jacks
      \\
      \SetCell[c=5]{c,bg=gray3,fg=white}\textbf{Main board, player side}
      \\
      \hspace{1em}
      & -- & Ribbon cable & 1
      & On the bottom-center of the board
      \\
      \hspace{1em}
      & VOL, GAIN  & A100k & 2
      & Potentiometers
      \\
      \hspace{1em}
      & BRIGHT & On-On & 1
      & Switch
      \\
      \hspace{1em}
      & CLIP & On-Off-On & 1
      & Switch
      \\
      \hspace{1em}
      & -- & LED & 1
      &
      \\
      \SetCell[c=5]{c,bg=gray3,fg=white}\textbf{Switch board, player side}
      \\
      \hspace{1em}
      & -- & Footswitch & 1
      & Switch
      \\
    \end{tblr}
  }
\end{table}

\restoregeometry{}

\subsection{Note on values}

Different kits and schematics designate values differently.
For example, these usually mean the same value:
\\
$\SI{2.2}{\kohm} = 2.2k = 2k2 = 2.2 \times 10^{3} Ohm = 2200Ohm$
\\
$\SI{4.7}{\uF} = 4.7u = 4u7 = 4.7 \times 10^{-6} Farad = 0.0000047 Farad$

\begin{table}[h!]
  \caption{Component values}
  \centerline{
    \begin{tblr}{
      hlines,
      vlines,
      rows={ht=1.2em},
      row{1}={bg=gray3,fg=white},
      colspec={Xrr}
    }
      \textbf{Value}
      & \textbf{Multiplier}
      & \textbf{Unit}
      \\
      \SetCell[c=3]{c}\textbf{Resistance}
      \\
      \SI{100}{\ohm}, 100R, 100 & 1 & Ohm
      \\
      \SI{1}{\kohm}, 1k & $10^{3}$ & Ohm
      \\
      \SI{1}{\Mohm}, 1M & $10^{6}$ & Ohm
      \\
      \SetCell[c=3]{c}\textbf{Capacitance}
      \\
      \SI{1}{\pF}, 1p & $10^{-12}$ & Farad
      \\
      \SI{1}{\nF}, 1n & $10^{-9}$ & Farad
      \\
      \SI{1}{\uF}, 1u & $10^{-6}$ & Farad
    \end{tblr}
  }
\end{table}

\end{document}
